\documentclass[12pt, titlepage]{article}

\usepackage{graphicx}
\usepackage{wrapfig}
\usepackage{hyperref}
\usepackage{siunitx}
\usepackage{amsmath}
\usepackage{enumerate}

\title{Flippin' Flingers Trebuchet Progress Report}
\author{Omar Ebrahim 110076575\\Saif Kaoud 110076323\\[10pt] Dr. John Magliaro\\
University of Windsor}

\begin{document}
    \maketitle
    \section{Abstract}
    This progress report provides an overview of the ongoing project focused on the sketching, designing, and modeling of a trebuchet.\\[10pt]
    The report outlines the key milestones achieved since the inception of the project, 
    including initial research on the mechanics of trebuchets, understanding the principles of their operation,
    and optimizing for distance.\\[10pt]
    Lastly, the report utilizes rough sketches, CAD, and numerical simulations
    to guide the construction of the final build.
    \newpage
    \tableofcontents
    \listoffigures
    \newpage
    \section{Introduction}
    This report's mission statement is to summarize the team's progress with:
    \begin{itemize}
        \item Preliminary design sketches.
        \item Detailed CAD drawings.
        \item Numerical modeling details and range predictions.
    \end{itemize}
    The problem outline is to design and analyze the trebuchet to maximize
    projectile distance and accuracy.\\[10pt]
    The trebuchet is designed for maximum distance through general
    plane motion.\footnote{Hibbeler, 2015}It is also designed for maximum 
    accuracy through string measure.\footnote{Rhoten, 2021}\\[10pt]
    To maximize distance, the velocity of the projectile is considered, as
    distance and velocity are proportional. Increasing the counterweight 
    fall distance can boost the projectile's velocity.\footnote{Siano, 2001}
    Increasing the counterweight-to-launch distance increases the velocity 
    of the projectile.\footnote{Denny, 2005}\\[10pt]
    Lastly, adjusting the trebuchet firing angle to 
    45 degrees achieves the maximum projectile distance.\footnote{Connel, 2001}
    \newpage
    \section{Methodology}
    The team starts by sketching the trebuchet on engineering grid 
    paper to create a rough visual representation.\\[10pt] 
    Next, the team uses CAD to design the trebuchet, incorporating
    precise dimensions and measurements. CAD also enables the team to view the trebuchet in 3D, aiding the real-life design process.\\[10pt]
    Finally, the team utilizes Working Model 2D, a CAE software, to simulate the trebuchet's motion and provide numerical data for analyzing the impact of various factors on its performance.
    \newpage
    \section{Results}
    \subsection{Sketch}
    Based on the provided information, the team chose to build a Floating Arm 
    Trebuchet, which is recognized for its unique feature of an axle allowing 
    unrestricted movement on the throwing arm. As a result, the trebuchet 
    exhibits general plane motion. The sketch is shown in Figure \ref{sketch}.\\[10pt]
    The frame of a trebuchet serves as the structural backbone that supports 
    and stabilizes the entire machine, and the arm is connected to the frame
    with wooden wheels.\\[10pt]
    The counterweight is released once the guide chute is triggered.
    When the counterweight is released, it falls, causing the throwing arm 
    to rotate rapidly around the axle. As the arm swings, the projectile 
    is released from the sling.
    \begin{figure}[t]                                  
    \centering
    \includegraphics[width=0.8\textwidth]{sketch.jpeg}
    \caption{Floating Arm Trebuchet sketch\label{sketch}}
    \end{figure}
    \newpage
    \subsection{Design}
    Figure \ref{CAD} showcases the CAD model of the Floating Arm Trebuchet. The design features a robust physical body, as depicted in the sketch. The counterweight is attachable to the middle axle, and the arm holds the sling and guide chute at the end. To enhance portability, wheels are attachable to the corners of the base, facilitating easy transportation of the trebuchet.\\[140pt]
    \textbf{Note:} The CAD file, as well as the Working Model 2D files, can be found in the submission of the progress report.   
    \begin{figure}[t]                                  
    \centering
    \includegraphics[width=0.8\textwidth]{CAD.png}
    \caption{Floating Arm Trebuchet in CAD\label{CAD}}
    \end{figure}
    \newpage
    \subsection{Model}
    The Floating Arm Trebuchet implemented in Working Model 2D can be found 
    in Figure \ref{model}. Additionally, a tracing of the trebuchet's motion 
    can be found in Figure \ref{motion}. Tracing is a feature in Working 
    Model 2D that shows each frame of the motion of an object.\\[10pt]
    Lastly, position-velocity-acceleration graphs for both vertical and 
    horizontal axes of the projectile are found in Figure \ref{graphs}.
    The horizontal velocity of the ball increases linearly once the trebuchet fires
    then stays constant (ignoring air resistance). Hence, the team predicts 
    with this model that they can achieve around 30 meters of projectile
    distance.

    \begin{figure}[t]                                  
    \centering
    \includegraphics[width=0.8\textwidth]{Model.png}
    \caption{Floating Arm Trebuchet model in Working Model 2D\label{model}}
    \end{figure}

\begin{figure}[b]
    \begin{minipage}[t]{0.69\textwidth}
        \vspace{12pt}
        \begin{flushleft}
            \includegraphics[width=\textwidth]{Graphs.png}
        \end{flushleft}
        \caption{P-V-A graph of ball\label{graphs}}
    \end{minipage}
    \hfill
    \begin{minipage}[t]{0.3\textwidth}
        \vspace{50pt}
        \begin{flushright}
            \includegraphics[width=\textwidth]{Motion.png}
        \end{flushright}
        \caption{Motion of Trebuchet\label{motion}}
    \end{minipage}
\end{figure}

    \newpage
    \section{Conclusions}
    The main objectives of this milestone were to create a blueprint for 
    the construction of the final build.
    The key findings are summarized as follows:
    \begin{enumerate}
        \item In order to maximize the distance traveled by the projectile, it is essential to optimize the projectile's velocity. 
        \item General plane motion increases projectile velocity by harnessing the speed of the falling counterweight.    
        \item Floating Arm Trebuchets consist of five main components: 
            frame, arm, counterweight, sling, and guide chute.
        \item The release of the projectile at a 45 degree angle achieves the 
            maximum projectile distance.
    \end{enumerate}
    \newpage
    \section{References}
        \hspace{15pt}Denny, M. (2005). Siege engine dynamics. European journal of physics, 26(4), 561.

        Hibbeler, R. C. (2015). 16.5. In Engineering mechanics: Dynamics (pp. 346–348). essay, Pearson.
        
        James O'Connell; Dynamics of a medieval missile launcher: the trebuchet. The Physics Teacher 1 November 2001; 39 (8): 471–473.

        Rhoten, R. P. (1999). The trebuchet: Accuracy analysis of a medieval siege engine. Volume 2: 19th Computers and Information in Engineering Conference.

        Siano, D. B. (2001). Trebuchet Mechanics. The Algorithmic Beauty of the Trebuchet.
\end{document}
