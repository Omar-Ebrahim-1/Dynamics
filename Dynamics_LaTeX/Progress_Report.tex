\documentclass[12pt, titlepage]{article}

\usepackage{graphicx}
\usepackage{wrapfig}
\usepackage{hyperref}
\usepackage{siunitx}
\usepackage{amsmath}
\usepackage{enumerate}

\title{Flippin' Flingers Trebuchet Progress Report}
\author{Omar Ebrahim 110076575\\Saif Kaoud 110076323\\[10pt] Dr. John Magliaro\\
University of Windsor}

\begin{document}
    \maketitle
    \section{Summary}
    \newpage
    \tableofcontents \newpage
    \listoffigures \newpage
    \listoftables \newpage
    \section{Introduction}
    This report's mission statement is to summarize the team's progress with:
    \begin{itemize}
        \item Preliminary design sketches.
        \item Detailed CAD drawings.
        \item Numerical modeling details and range predictions.
    \end{itemize}
    The problem outline is to design and analyze the trebuchet to maximize
    projectile distance and accuracy.\\[10pt]
    The trebuchet is designed for maximum distance through general
    plane motion.\footnote{Hibbeler, 2015}It is also designed for maximum 
    accuracy through string measure.\footnote{Rhoten, 2021}\\[10pt]
    To maximize distance, the velocity of the projectile is considered, as
    distance and velocity are proportional. Increasing the counterweight 
    fall distance can boost the projectile's velocity.\footnote{Siano, 2001}
    Increasing the counterweight-to-launch distance increases the velocity 
    of the projectile.\footnote{Denny, 2005}\\[10pt]
    Lastly, adjusting the trebuchet firing angle to 
    45 degrees achieves the maximum projectile distance.\footnote{Connel, 2001}
    \newpage
    \section{Main}
    \newpage
    \section{Conclusions}
    The main objectives of this milestone were [complete the statement].
    The key findings are summarized as follows:
    \begin{enumerate}
        \item l
    \end{enumerate}
    \newpage
    \section{References}
        \hspace{15pt}Hibbeler, R. C. (2015). 16.5. In Engineering mechanics: Dynamics (pp. 346–348). essay, Pearson.

        Rhoten, R. P. (1999). The trebuchet: Accuracy analysis of a medieval siege engine. Volume 2: 19th Computers and Information in Engineering Conference.

        Siano, D. B. (2001). Trebuchet Mechanics. The Algorithmic Beauty of the Trebuchet.

        Denny, M. (2005). Siege engine dynamics. European journal of physics, 26(4), 561.

        James O'Connell; Dynamics of a medieval missile launcher: the trebuchet. The Physics
        Teacher 1 November 2001; 39 (8): 471–473.
    \newpage

    \appendix
    \section{Appendix Stuff}
\end{document}
